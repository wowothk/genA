\documentclass{article}
\usepackage{graphicx,hyperref,url}
\usepackage{amsmath}
\renewcommand{\figurename}{Gambar}
\renewcommand{\tablename}{Tabel}
\newcommand{\comment}[1]{}
\usepackage{rotating}
\begin{document}
	\section*{Data Kapasitas dan Distribusi Semen Tonasa}
	\begin{table}[h]
		\caption{Unit Biaya Bahan Baku}
		\begin{tabular}{|c|r|r|r|r|r|}
			\hline
			\multicolumn{ 1}{|c|}{Id} &   \multicolumn{ 5}{|c|}{Unit biaya bahan baku dari pemasok(Rp)} \\

			\multicolumn{ 1}{|c|}{} &       2011 &       2012 &       2013 &       2014 &       2015 \\
			\hline
			1 & 67.919.188.889 & 79.813.778.052 & 132.038.324.222 & 151.794.605.175 & 97.162.134.888 \\
			\hline
		\end{tabular}
	\end{table}  
	\begin{table}[h]
		\caption{Kapasitas Produksi Pabrik}
		\begin{tabular}{|c|c|rrrrr|}
			\hline
			\multicolumn{ 1}{|c|}{Id Pabrik} & \multicolumn{ 1}{|c|}{Nama Pabrik} &                   \multicolumn{ 5}{|c}{Kapasitas Pabrik (ton)} \\
			
			\multicolumn{ 1}{|c|}{} & \multicolumn{ 1}{|c|}{} &       2011 &       2012 &       2013 &       2014 &       2015 \\
			\hline
			1 &   Tonasa 2 &    590.000 &    590.000 &    590.000 &    590.000 &    590.000 \\
			\hline
			2 &   Tonasa 3 &    590.000 &    590.000 &    590.000 &    590.000 &    590.000 \\
			\hline
			3 &   Tonasa 4 &  2.300.000 &  2.300.000 &  2.300.000 &  2.300.000 &  2.300.000 \\
			\hline
			4 &   Tonasa 5 &          - &  2.500.000 &  2.500.000 &  2.500.000 &  2.500.000 \\
			\hline
		\end{tabular}  
	\end{table}
\begin{table}[h]
	% Table generated by Excel2LaTeX from sheet 'kapsitas pengantongan'
	\caption{Kapasitas Unit Pengantongan}
	\begin{tabular}{|r|r|r|}
		\hline
		Id Unit Pengantongan & Nama Unit Pengantongan & Kapasitas Unit Pengantongan (ton/tahun) \\
		\hline
		1 & Pengantongan biringkassi &  2.500.000 \\
		\hline
		2 & Pengantongan makassar &    600.000 \\
		\hline
	\end{tabular}  
\end{table}

\begin{table}[h]
	\caption{jumlah distributor}
	% Table generated by Excel2LaTeX from sheet 'jumlah distributor'
	\begin{tabular}{|c|ccccc|}
		\hline
		\multicolumn{ 1}{|c|}{Id} &                       \multicolumn{ 5}{|c}{Jumlah distributor} \\
		
		\multicolumn{ 1}{|c|}{} &       2011 &       2012 &       2013 &       2014 &       2015 \\
		\hline
		1 &         28 &         28 &         28 &         28 &         28 \\
		\hline
	\end{tabular}  
	
\end{table}

\begin{table}[h]
	\caption{Jumlah Bahan Baku}
	% Table generated by Excel2LaTeX from sheet 'data jumlah bahan baku'
	\begin{tabular}{|c|c|c|c|c|c|c|}
		\hline
		\multicolumn{ 1}{|c|}{Id} & \multicolumn{ 1}{|c|}{Nama Pabrik} & \multicolumn{ 5}{|c|}{jumlah bahan baku dari pemasok untuk pabrik (ton)} \\
		\hline
		\multicolumn{ 1}{|c|}{} & \multicolumn{ 1}{|c|}{} &       2011 &       2012 &       2013 &       2014 &       2015 \\
		\hline
		1 &   Tonasa 2 &    154.270 &    137.654 &    187.296 &    192.636 &    121.384 \\
		\hline
		2 &   Tonasa 3 &    160.543 &    153.586 &    173.608 &    207.972 &    128.768 \\
		\hline
		3 &   Tonasa 4 &    579.503 &    572.652 &    713.472 &    762.685 &    449.915 \\
		\hline
		4 &   Tonasa 5 &          - &    144.906 &    621.906 &    709.776 &    512.253 \\
		\hline
	\end{tabular}  
\end{table}

\begin{table}[h]
	\caption{data distribusi dari pabrik ke unit pengantongan}	
	% Table generated by Excel2LaTeX from sheet 'pabrik ke pengantongan'
	\begin{tabular}{|r|c|c|c|c|c|c|c|}
		\hline
		\multicolumn{ 1}{|r|}{Id} & \multicolumn{ 1}{|c|}{Nama Pabrik} & \multicolumn{ 1}{|c|}{Nama Unit Pengantongan} & \multicolumn{ 5}{|c|}{Jumlah produk dari pabrik ke unit pengantongan (ton)} \\
		\hline
		\multicolumn{ 1}{|r|}{} & \multicolumn{ 1}{|c|}{} & \multicolumn{ 1}{|c|}{} &       2011 &       2012 &       2013 &       2014 &       2015 \\
		\hline
		\multicolumn{ 1}{|r|}{1} & \multicolumn{ 1}{|c|}{Tonasa 2} & UP Biringkassi &     65.434 &     62.648 &     72.967 &     74.841 &     69.553 \\
		\hline
		\multicolumn{ 1}{|r|}{} & \multicolumn{ 1}{|c|}{} &        UP  & \multicolumn{ 1}{|c|}{14.623} & \multicolumn{ 1}{|c|}{14.000} & \multicolumn{ 1}{|c|}{16.306} & \multicolumn{ 1}{|c|}{16.725} & \multicolumn{ 1}{|c|}{15.544} \\
		\hline
		\multicolumn{ 1}{|r|}{} & \multicolumn{ 1}{|c|}{} &   Makassar & \multicolumn{ 1}{|c|}{} & \multicolumn{ 1}{|c|}{} & \multicolumn{ 1}{|c|}{} & \multicolumn{ 1}{|c|}{} & \multicolumn{ 1}{|c|}{} \\
		\hline
		\multicolumn{ 1}{|r|}{2} & \multicolumn{ 1}{|c|}{Tonasa 3} & UP Biringkassi &     68.095 &     69.898 &     67.635 &     80.799 &     73.784 \\
		\hline
		\multicolumn{ 1}{|r|}{} & \multicolumn{ 1}{|c|}{} &        UP  & \multicolumn{ 1}{|c|}{15.218} & \multicolumn{ 1}{|c|}{15.621} & \multicolumn{ 1}{|c|}{15.115} & \multicolumn{ 1}{|c|}{18.057} & \multicolumn{ 1}{|c|}{16.489} \\
		\hline
		\multicolumn{ 1}{|r|}{} & \multicolumn{ 1}{|c|}{} &   Makassar & \multicolumn{ 1}{|c|}{} & \multicolumn{ 1}{|c|}{} & \multicolumn{ 1}{|c|}{} & \multicolumn{ 1}{|c|}{} & \multicolumn{ 1}{|c|}{} \\
		\hline
		\multicolumn{ 1}{|r|}{3} & \multicolumn{ 1}{|c|}{Tonasa 4} & UP Biringkassi &    245.798 &    260.619 &    277.956 &    296.311 &    257.802 \\
		\hline
		\multicolumn{ 1}{|r|}{} & \multicolumn{ 1}{|c|}{} &        UP  & \multicolumn{ 1}{|c|}{54.930} & \multicolumn{ 1}{|c|}{58.242} & \multicolumn{ 1}{|c|}{62.117} & \multicolumn{ 1}{|c|}{66.218} & \multicolumn{ 1}{|c|}{57.613} \\
		\hline
		\multicolumn{ 1}{|r|}{} & \multicolumn{ 1}{|c|}{} &   Makassar & \multicolumn{ 1}{|c|}{} & \multicolumn{ 1}{|c|}{} & \multicolumn{ 1}{|c|}{} & \multicolumn{ 1}{|c|}{} & \multicolumn{ 1}{|c|}{} \\
		\hline
		\multicolumn{ 1}{|r|}{4} & \multicolumn{ 1}{|c|}{Tonasa 5} & UP Biringkassi &          - &     65.948 &    242.284 &    275.755 &    293.522 \\
		\hline
		\multicolumn{ 1}{|r|}{} & \multicolumn{ 1}{|c|}{} &        UP  & \multicolumn{ 1}{|c|}{-} & \multicolumn{ 1}{|c|}{14.738} & \multicolumn{ 1}{|c|}{54.145} & \multicolumn{ 1}{|c|}{61.625} & \multicolumn{ 1}{|c|}{65.595} \\
		\hline
		\multicolumn{ 1}{|r|}{} & \multicolumn{ 1}{|c|}{} &   Makassar & \multicolumn{ 1}{|c|}{} & \multicolumn{ 1}{|c|}{} & \multicolumn{ 1}{|c|}{} & \multicolumn{ 1}{|c|}{} & \multicolumn{ 1}{|c|}{} \\
		\hline
	\end{tabular}  
\end{table}

\begin{table}[h]
	\caption{Jumlah Permintaan (dari pengantongan ke distributor)}
	% Table generated by Excel2LaTeX from sheet 'pengantongan ke distributor'
	\begin{tabular}{|c|r|c|c|c|c|c|c|c|c|c|c|}
		\hline
		\multicolumn{ 1}{|c|}{Id} & \multicolumn{ 1}{|c|}{Nama Distributor} &                                                    \multicolumn{ 10}{|c|}{jumlah permintaan produk dari unit pengantongan(ton)} \\
		\hline
		\multicolumn{ 1}{|c|}{} & \multicolumn{ 1}{|c|}{} & \multicolumn{ 2}{|c|}{2011} & \multicolumn{ 2}{|c|}{2012} & \multicolumn{ 2}{|c|}{2013} & \multicolumn{ 2}{|c|}{2014} & \multicolumn{ 2}{|c|}{2015} \\
		\hline
		\multicolumn{ 1}{|c|}{} & \multicolumn{ 1}{|c|}{} & Biringkassi &   Makassar & Biringkassi &   Makassar & Biringkassi &   Makassar & Biringkassi &   Makassar & Biringkassi &   Makassar \\
		\hline
		1 & CV. Karya Baru &       7090 &      10000 &        630 &      17100 &       1890 &      30000 &       1001 &      37700 &       2000 &      29800 \\
		\hline
		2 & Kokpar Semen Tonasa &      10000 &       5201 &      20000 &        200 &      30000 &       2744 &      30000 &       9870 &      30000 &       9800 \\
		\hline
		3 & PT. Mega Indah Sari Timor &       6200 &      14000 &       1300 &      15000 &       9200 &      28790 &       2800 &      20000 &       8900 &      20000 \\
		\hline
		4 & PT. Mitra Pembangunan Nusantara &      11000 &        511 &      20000 &       9385 &      20000 &       4870 &      30000 &        800 &      20000 &       9870 \\
		\hline
		5 & UD. Multi Guna Rejeki &        300 &      10000 &       7310 &      10000 &        890 &      30000 &       1600 &      29000 &       9100 &      20000 \\
		\hline
		6 & PT. Prima Karya Manunggal &      12000 &       1211 &      10000 &        100 &      29000 &       9900 &      30000 &       9000 &      20000 &       2000 \\
		\hline
		7 & PT. Padi Mas Prima &        100 &      11100 &        600 &      15000 &        700 &      20000 &        200 &      20000 &        500 &      30000 \\
		\hline
		8 & CV. Bintang Mas Jaya &        800 &      15110 &       4000 &      19000 &       4500 &      29700 &       5000 &      29840 &        500 &      28900 \\
		\hline
		9 & PT. Catut Kencana Sakti &        250 &      18300 &       5000 &      20200 &       4500 &      32100 &       5000 &      32900 &      10000 &      31002 \\
		\hline
		10 & CV. Gajaco Utama &        250 &      14000 &       5000 &      17100 &      10000 &      35770 &      10000 &      29800 &      10000 &      29800 \\
		\hline
		11 & CV. Ichal Mutiara &        500 &        600 &      10000 &        500 &      10000 &        400 &      10000 &       2000 &      10000 &       9780 \\
		\hline
		12 & CV. Indah Sari &        203 &      13100 &        200 &      15000 &       4600 &      28000 &        700 &      30000 &       1000 &      20000 \\
		\hline
		13 & UD. Pelita Indah – Mks &        800 &      10000 &      10000 &      20100 &      15000 &      29650 &       5000 &      31800 &      15000 &      29000 \\
		\hline
		14 & UD. Pertama &        800 &       6560 &      10000 &       8000 &      15000 &       1900 &      30000 &       1900 &      15000 &       9000 \\
		\hline
		15 & PT. Rajawali Jaya Sakti &        890 &      14000 &       1600 &      10000 &       9800 &      30000 &       2000 &      30000 &       7890 &      20000 \\
		\hline
		16 & CV. Sinar Dagang &      16000 &        322 &      20000 &        400 &      20000 &        900 &      30000 &        900 &      20000 &        200 \\
		\hline
		17 & UD. Tirograha Niaga &      21200 &        100 &      22000 &       6000 &      30900 &       2500 &      29800 &      10000 &      29160 &       1000 \\
		\hline
		18 & PT. HS. Bangunan &      17230 &       1300 &       9500 &      10000 &      32900 &       2500 &      32900 &      10000 &      29000 &      15000 \\
		\hline
		19 & Perusda Perdagangan Umum &        891 &      24000 &       9500 &      10000 &       1020 &      30000 &       1000 &      10000 &        900 &      15000 \\
		\hline
		20 & Perusahaan Daerah Bajiminasa &       7000 &      20000 &        200 &      23100 &      15000 &      38900 &      20000 &      35000 &      15000 &      32800 \\
		\hline
		21 & PT. Ta’disangka &      10000 &        710 &      23000 &       1200 &      15000 &       9800 &      10000 &       2900 &      15000 &       1900 \\
		\hline
		22 & UD. Jaya Mandiri &       3500 &      16000 &        300 &      20000 &       1500 &      20000 &        900 &      30000 &        800 &      30000 \\
		\hline
		23 & Kopeerasi Perdagangan Indonesia &       3500 &      12100 &       1000 &      18600 &       1500 &      29800 &       1000 &      31900 &       4000 &      29100 \\
		\hline
		24 & CV. Semento Pratama &       2000 &      14590 &       1000 &      19200 &       6000 &      28700 &       1000 &      31000 &       4000 &      33200 \\
		\hline
		25 & Sun Lik Internasional Energy &       2000 &      15610 &       5000 &      19200 &       6000 &      29000 &       5000 &      32100 &       5000 &      33200 \\
		\hline
		26 & CV. Empos Tiran &       2000 &      18900 &       5000 &      22000 &       6000 &      31900 &       5000 &      31000 &       5000 &      29800 \\
		\hline
		27 & Perusahaan Daerah Gowa Mandidri &       2000 &      20000 &       5000 &      20000 &       6000 &      20000 &      10000 &      30000 &       5000 &      30000 \\
		\hline
		28 & Rinzhu Lalogau Engineriing Contruction &       2000 &        670 &       5000 &       3190 &       6000 &       9800 &      10000 &       1000 &       5000 &       2000 \\
		\hline
	\end{tabular}  
	
\end{table}


\begin{table}[h]
	\caption{Biaya distribusi dari pemasok ke pabrik}
	% Table generated by Excel2LaTeX from sheet 'biaya pemasok ke pabrik'
	\begin{tabular}{|c|c|c|c|c|c|}
		\hline
		\multicolumn{ 1}{|c|}{Node} &   \multicolumn{ 5}{|c|}{Biaya dari pemasok ke pabrik (Rp/ton)} \\
		\hline
		\multicolumn{ 1}{|c|}{} &       2011 &       2012 &       2013 &       2014 &       2015 \\
		\hline
		1 &     12.500 &     13.300 &     13.300 &     14.000 &     14.000 \\
		\hline
	\end{tabular}  

\end{table}
\begin{table}[h]
	\caption{biaya distribusi dari pabrik ke unit pengantongan}
	% Table generated by Excel2LaTeX from sheet 'biaya pabrik ke pengantongan'
	\begin{tabular}{|c|r|r|r|r|r|}
		\hline
		\multicolumn{ 1}{|c|}{Nama pabrik} & \multicolumn{ 5}{|c|}{Biaya dari pabrik ke unit pengantongan (Rp)} \\
		\hline
		\multicolumn{ 1}{|c|}{} &       2011 &       2012 &       2013 &       2014 &       2015 \\
		\hline
		Tonasa 2 & 14.433.271.144 & 21.297.146.667 & 10.843.045.945 & 19.061.058.501 & 9.435.098.393 \\
		\hline
		Tonasa 3 & 15.020.144.247 & 23.762.003.640 & 10.050.596.401 & 20.578.576.658 & 10.009.029.431 \\
		\hline
		Tonasa 4 & 54.217.326.172 & 88.597.803.101 & 41.304.693.134 & 75.466.620.566 & 34.971.620.335 \\
		\hline
		Tonasa 5 &          - & 22.419.191.132 & 36.003.742.042 & 70.231.327.736 & 39.817.108.154 \\
		\hline
	\end{tabular}  	
\end{table}
\begin{table}
	\caption{biaya distribusi dari pengantongan ke distributor}
	% Table generated by Excel2LaTeX from sheet 'biaya pengantongan ke dist'
	\begin{tabular}{|c|c|rrrrr|}
		\hline
		\multicolumn{ 1}{|c|}{Id} & \multicolumn{ 1}{|c|}{Unit Pengantongan} & \multicolumn{ 5}{|c}{Biaya dari unit pengantongan ke distributor(Rp)} \\
		
		\multicolumn{ 1}{|c|}{} & \multicolumn{ 1}{|c|}{} &       2011 &       2012 &       2013 &       2014 &       2015 \\
		\hline
		1 & UP Biringkassi & 40.083.437.689 & 55.600.723.524 & 74.745.895.781 & 77.385.383.908 & 74.386.025.177 \\
		\hline
		2 & UP Makassar & 8.957.694.971 & 12.425.439.289 & 16.703.929.935 & 17.293.792.754 & 16.623.507.415 \\
		\hline
	\end{tabular}  
\end{table}

\end{document}